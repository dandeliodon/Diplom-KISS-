%
% File: introduction.tex
% Author: Tengxiang Li
%
\chapter{Introduction}
\noindent\hspace{2em}In today’s digital era, organizations face escalating cybersecurity threats and increasingly stringent regulatory requirements for protecting information. Businesses and governments worldwide have turned to established security standards and frameworks to manage cyber risks systematically. In Kazakhstan, like in many countries, there is a growing need to harmonize global best practices with local laws to safeguard data and ensure compliance. Astana IT University’s interest in this topic reflects the national priority on strengthening information security, as critical sectors (e.g. finance, government services) digitalize and integrate with global networks.
This thesis titled “Development and Implementation of Security Standards and Policies for Organizations” examines how organizations can effectively implement information security standards and policies. It addresses both the theoretical frameworks and the practical tools needed to enforce those standards. The research is motivated by challenges observed in aligning international cybersecurity standards – such as ISO/IEC 27001, NIST CSF, and GDPR – with Kazakhstan’s legal and operational context. Although frameworks like ISO 27001 are internationally recognized for improving an organization’s security management, their adoption must consider national legislation (for example, Kazakhstan’s Law “On Informatization” and the Law “On Personal Data and Its Protection”). Organizations in Kazakhstan, especially in regulated industries like banking, must navigate requirements from the National Bank and other authorities while also aspiring to global best practices.
Research Objectives: The primary objectives of this thesis are: (1) to perform a comparative analysis of key international cybersecurity standards (ISO/IEC 27001:2022, NIST CSF 2.0, and GDPR) and determine how they can be adapted to Kazakhstan’s context; (2) to develop a conceptual framework and prototype tools (web and mobile applications) that assist organizations in implementing these standards and crafting compliant security policies; (3) to create a set of original security policies and controls for a model Kazakhstani organization, integrating international standards with local regulatory compliance; and (4) to evaluate the implementation through a fictional use-case (QazFinTech Bank) and discuss the outcomes, benefits, and challenges.
Significance: By addressing these objectives, the thesis aims to provide both scholarly insight and practical solutions. From an academic perspective, it contributes to understanding the interplay between global cybersecurity frameworks and domestic regulations. From a practical standpoint, it delivers a prototype “toolkit” – a website and mobile app – that could be further developed to help real organizations in Kazakhstan manage their information security programs. The fictional case study of QazFinTech Bank is used as a realistic scenario to ground the discussion in practical application. Ultimately, this research supports Kazakhstani organizations in bolstering their cybersecurity maturity in an era of international data exchange and sophisticated cyber threats.
The remainder of this thesis is structured as follows: the Problem Statement defines the core problems and gaps this research addresses. The Literature Review surveys relevant standards and laws, comparing their content and implications. The Methodology chapter explains the research approach, including how the security framework integration and software development were carried out. The Practical Implementation chapter describes the developed web and mobile solutions and how they function in the case study scenario. The Development of Security
Standards and Policies chapter presents the custom-designed policies, standards, and control mappings for the fictional organization. The Results and Discussion chapter evaluates how the proposed standards and tools would operate in practice, examining benefits, limitations, and compliance outcomes. Finally, the Conclusion and Future Work chapter summarizes key findings, recommends how these efforts could be scaled nationally, and suggests directions for future research.


%
%

