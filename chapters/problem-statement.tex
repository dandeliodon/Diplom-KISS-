\chapter{Problem Statement}
\addcontentsline{toc}{chapter}{Problem Statement}

\noindent\hspace{2em}Organizations in Kazakhstan face a dual challenge in information security management: they must adhere to international best practices to effectively mitigate cyber risks, while also complying with local regulatory requirements. This dual mandate can be problematic due to gaps between global frameworks and local laws, and a lack of readily available guidance on integrating the two. Key problems addressed in this thesis include:

- \textbf{Lack of Integrated Security Frameworks.} Many Kazakhstani organizations do not have a comprehensive information security management system (ISMS) in place. For instance, the National Bank of Kazakhstan noted the need for banks to establish information security management systems and formal security procedures. However, there is uncertainty about which framework to adopt (e.g., ISO vs. NIST) and how to align it with Kazakhstan’s context. This results in inconsistent security postures and difficulties in demonstrating compliance.

- \textbf{Alignment with International Standards.} Frameworks like ISO/IEC 27001 and NIST CSF offer structured approaches to security. Yet, adoption of these standards in Kazakhstan has been limited, partly due to a lack of localized guidance. The question arises: how can ISO 27001:2022 and NIST CSF 2.0 be adapted to the legal, cultural, and operational environment of Kazakhstani organizations?

- \textbf{Compliance with Data Protection Regulations.} Kazakhstan’s Law “On Personal Data and Its Protection” (2013) imposes requirements on data handling. Organizations need to align internal policies with GDPR-like principles (consent, minimization, breach notification), while also respecting local mandates such as data localization.

- \textbf{Operational Gap – Lack of Tools for Implementation.} Even when policies exist, many organizations lack systems to enforce them. Employees may not be aware, controls may be inconsistently applied, and audits reveal gaps. A centralized tool for policy enforcement, incident response, training, and compliance tracking is lacking.

\textbf{Fictional Case Focus – QazFinTech Bank.}  
This thesis uses a fictional mid-size Kazakhstani financial organization to examine how ISO 27001, NIST CSF, and GDPR principles can be integrated into a unified security policy framework adapted to national law. It explores what a practical implementation tool for this integration might look like.

\textbf{Summary.}  
The core problem is bridging the gap between international cybersecurity frameworks and local organizational practice in Kazakhstan. This includes aligning standards with laws, developing policy, and applying them through enforceable tools to enhance resilience and compliance.
